\documentclass[12pt]{article}
\usepackage{setspace} % double space
\onehalfspacing % 1.5 space
\usepackage[left=1in,top=1in,right=1in,bottom=1in,nohead,nofoot]{geometry} %margins
\setlength\footskip{0.25in} % separate page number from last line
\setlength\parindent{0.5in} % indent paragraphs; first paragraphs are not indented
\usepackage{epstopdf}
\usepackage{graphicx} % Use for graphics

% begin upright mu (\umu) definition
%\usepackage{lmodern}
%\usepackage{textcomp}
%\DeclareFontFamily{U}{euc}{}
%\DeclareFontShape{U}{euc}{m}{n}{<-6>eurm5<6-8>eurm7<8->eurm10}{}
%\DeclareSymbolFont{AMSc}{U}{euc}{m}{n}
%\DeclareMathSymbol{\umu}{\mathord}{AMSc}{"16}
% end upright mu


\usepackage{amssymb}
\usepackage{amsmath}
%\usepackage{MnSymbol} % this must be defined after \umu (upright mu)

\begin{document}

\begin{center}
{\LARGE 
Orthogonality of Modulations of the Unpolarized Dihadron Cross Section
}
\end{center}


\section*{Motivation}

Let $\{\alpha_k(\phi_h,\phi_R,\theta)\}$ and $\{\beta_k(\phi_h,\phi_R,\theta)\}$
be sets of modulations of $d\sigma_{LU}$ and $d\sigma_{UU}$, respectively, and
with $k$ an index. These sets are subsets of the Fourier series basis set, and
thus are assumed to be mutually orthogonal, however, with acceptance
limitations, inner product integration ranges are limited and full mutual
orthogonality is not guaranteed in a measurement. In the following, associate
each modulation to a Hilbert space vector and adopt bra-ket notation for inner
products. The differential cross sections are:
\begin{align}
\displaystyle
&|d\sigma_{LU}\rangle=\sum_{i}{a_i|\alpha_i\rangle}~ ~\text{where}~ ~
a_i=K_A(x,y)F_{LU}^{\alpha_i}\\
&|d\sigma_{UU}\rangle=\sum_{i}{b_i|\beta_i\rangle}~ ~\text{where}~ ~
b_i=K_B(x,y)F_{UU}^{\beta_i}
\end{align}
$K$ denotes a kinematic depolarization factor, and $F$ denotes a structure function.

The beam spin asymmetry modulated by $\alpha_k$ is
\begin{equation}
\displaystyle
A_{LU}^{\alpha_k}=\frac{\langle\alpha_k|d\sigma_{LU}\rangle}
                  {\langle1|d\sigma_{UU}\rangle}
\end{equation}
that is, the ratio of the $\alpha_k$-moment of $d\sigma_{LU}$ to the
constant moment, {\it i.e.}, the total integral of $d\sigma_{UU}$, denoted
$\sigma$. Letting $|\beta_0\rangle=|1\rangle$ and assuming normalization
$\langle\alpha_i|\alpha_i\rangle=\langle\beta_i|\beta_i\rangle=1$, the asymmetry is 
\begin{equation}
A_{LU}^{\alpha_k}=\dfrac
{ a_k+\sum_{i\neq k}{\langle \alpha_k|\alpha_i\rangle a_i} }
{ b_0+\sum_{i\neq 0}{\langle 1|\beta_i\rangle b_i} }
\end{equation}
If all modulations were mutually orthogonal, then $A_{LU}^{\alpha_k}=a_k/b_0$,
and we can associate $b_0=\sigma$. Since they are not orthogonal, define
$A_k=a_k/\sigma$ and $B_k=b_k/\sigma$ so that
\begin{equation}
A_{LU}^{\alpha_k}=\dfrac
{ A_k+\sum_{i\neq k}{\langle \alpha_k|\alpha_i\rangle A_i} }
{ 1+\sum_{i\neq 0}{\langle 1|\beta_i\rangle B_i} }
\end{equation}
This equation represents the general fit form for the beam spin asymmetry.


\section*{Orthogonality}

We assess the orthogonality of the various modulations of the unpolarized
dihadron cross section, weighted by the acceptance at CLAS. This is in part an
answer to Alessandro's question asked at CPHI, regarding the necessity of including
$\phi$-dependence in the denominator of the beam spin asymmetry.

The unpolarized
cross section contains three structure functions at twists 2 and 3, each of
which involve the convolution of PDFs with DiFFs; the DiFFs can be expanded in
partial waves, denoted by the angular momentum state $|\ell,m\rangle$. The
structure functions are:
\begin{align}
&F_{UU,T}^{P_{\ell,m}\cos\left(m\phi_h-m\phi_R\right)}
  \sim f_1\otimes D_1^{|\ell,m\rangle}\\
&F_{UU}^{P_{\ell,m}\cos\left[(2-m)\phi_h+m\phi_R\right]}
  \sim h_1^{\perp}\otimes H_1^{\perp|\ell,m\rangle}\\
&F_{UU}^{P_{\ell,m}\cos\left[(1-m)\phi_h+m\phi_R\right]}
  \sim h\otimes H_1^{\perp|\ell,m\rangle}+
       f^{\perp}\otimes D_1^{|\ell,m\rangle}
\end{align}
The subscripts denote the polarization, where $UU$ means unpolarized beam and
target; if there is a third subscript, it denotes the polarization of the
virtual photon. The superscripts denote the modulation function, dependent on
$\phi_h,\phi_R,\theta$, and $P_{\ell,m}$ represents associated Legendre
polynomials of $\cos\theta$. The first two structure functions are twist-2, and
the third is twist-3, and for brevity the twist-3 structure function omits the
terms containing twist-3 DiFFs as well as the relative kinematic factors between
the two written terms. Finally, to simplify notation, we denote each modulation
by the associated state vector $|\ell,m>_{polarization}^{twist}$, and we
truncate the partial wave expansions at $\ell_{\text{max}}=2$.

The denominator of an asymmetry involves an integral of the cross section
$d\sigma_{UU}$, that is, the so-called ``constant moment'' of $d\sigma_{UU}$.
Ideally, all terms except for those which are not dependent on
$\phi_h,\phi_R,\theta$ vanish under the integral; the only modulation which is
constant and independent of these variables is $|0,0\rangle_{UU,T}^{tw2}$. Because the
usual orthogonality weight for the $\{P_{\ell,m}\}$ set is typically
omitted\footnote{which also causes the $ss$ partial wave $|0,0\rangle$ to contain
contamination from a pure $p$-wave}, the $|2,0\rangle_{UU,T}^{tw2}$ term also
survives the integration. Contributions from this extra term have been accounted
for as a systematic uncertainty, following the methodology developed by HERMES;
the technique applies positivity bounds of the DiFFs, as well as the fraction of
inclusive dihadrons which originate from vector meson decay.

The arguments regarding the surviving terms of the integration are predicated on
the orthogonality of $|0,0\rangle_{UU,T}^{tw2}$ between all other
modulations\footnote{except for $|2,0\rangle_{UU,T}^{tw2}$, of course}. With the
acceptance limitations, this is not necessarily true. Figure \ref{innerproducts}
shows the CLAS acceptance-weighted inner products of $|0,0\rangle_{UU,T}^{tw2}$
between all other relevant modulations. The inner products are normalized, so
that the inner product of $|0,0\rangle_{UU,T}^{tw2}$ with itself is 1. Clearly
the inner product with $|2,0\rangle_{UU,T}^{tw2}$ is large, for the
aforementioned reasons; however, several other inner products are also quite
large.


\begin{figure}[h!]
\centering
\includegraphics[width=0.2\textwidth]{./matCanv1.png}
\includegraphics[width=0.3\textwidth]{./matCanv2.png}
\caption{Acceptance-weighted inner products of modulations with $|0,0\rangle_{UU,T}^{tw2}$}
\label{innerproducts}
\end{figure}

We need to assess how these non-orthogonal modulations impact the beam spin
asymmetry, especially if the corresponding PDFs and DiFFs, such as the
Boer-Mulders function $h_1^{\perp}$, are large in our kinematic bins. I am not
yet sure how to do this.

It should be noted that Stefan has studied the inclusion of denominator
modulation amplitudes in the single-hadron beam spin asymmetry, and he does
observe an effect which is stronger at high $p_T$. Applying his acceptance
function and repeating the above calculation of inner products for the
single-hadron unpolarized cross section modulations also yields large inner
products.  Moreover, the inner products are much larger at high $p_T$ than at
low $p_T$, in agreement with his observation. It is thus worthy to repeat this
evaluation for dihadrons in different kinematic bins. 




\end{document} 
