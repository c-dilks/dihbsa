\documentclass[12pt]{article}
\usepackage{setspace} % double space
\onehalfspacing % 1.5 space
\usepackage[left=1in,top=1in,right=1in,bottom=1in,nohead,nofoot]{geometry} %margins
\setlength\footskip{0.25in} % separate page number from last line
\setlength\parindent{0.5in} % indent paragraphs; first paragraphs are not indented
\usepackage{epstopdf}

% begin upright mu (\umu) definition
%\usepackage{lmodern}
%\usepackage{textcomp}
%\DeclareFontFamily{U}{euc}{}
%\DeclareFontShape{U}{euc}{m}{n}{<-6>eurm5<6-8>eurm7<8->eurm10}{}
%\DeclareSymbolFont{AMSc}{U}{euc}{m}{n}
%\DeclareMathSymbol{\umu}{\mathord}{AMSc}{"16}
% end upright mu


\usepackage{amssymb}
%\usepackage{MnSymbol} % this must be defined after \umu (upright mu)

\begin{document}

\begin{center}
{\LARGE 
Orthonormality of Modulations
}
\end{center}

\subsection*{Testing Orthonormality under Acceptance Constraints}


The goal is to assess the orthogonality of the various modulations of the beam
spin asymmetry; let $S$ be the set of such modulation functions, dependent on
$\phi_h$, $\phi_R$, and $\theta$. The functions in $S$ are preferred to be
mutually orthogonal, since the measurement of an amplitude of one modulation $f$
can be impacted by the amplitudes of other modulations which are not orthogonal
to $f$.  If the yield data do not cover the full $\phi_h,\phi_R,\theta$ range,
the orthogonality of the functions of $S$ can be impacted, as the integration
ranges will be limited by the acceptance. 

The functions of $S$ are written as the product of two functions, 
$P_{\ell}^m\Phi_t^m$, which depend
on the angular momentum state $|\ell,m\rangle_t$ of the dihadron system, where
$\ell$ and $m$ are the angular momentum eigenvalues and $t$ is the twist:
\begin{enumerate}
\item $\Phi_t^m\left(\phi_h,\phi_R\right)$: 
azimuthal modulations, which depend on $m$ and twist $t$; note that $\Phi_2^0=0$ and
$\Phi_2^{-m}=-\Phi_2^{m}$.
  \begin{itemize}
  \item twist 2: $\Phi_2^m=\sin\left[m\left(\phi_h-\phi_R\right)\right]$
  \item twist 3: $\Phi_3^m=\sin\left[\left(1-m\right)\phi_h+m\phi_R\right]$
  \end{itemize}
\item $P_{\ell}^{m}\left(\cos\theta\right)$:
associated Legendre polynomials, which depend on the angular momentum
eigenstate; note that $P_{\ell}^{-m}=P_{\ell}^{m}$ is assumed. This is the
partial wave expansion of the DiFFs in $\cos\theta$.
\end{enumerate}
The functions for $\ell\in\{0,1,2\}$ are (up to normalization factors):
\begin{center}
\begin{tabular}{|c|c|c|}
\hline
$|\ell,m\rangle$ & twist-2 & twist-3 \\\hline\hline
$|0,0\rangle$ & 0 & $\sin\phi_h$ \\\hline
\end{tabular}
~ ~ ~
\begin{tabular}{|c|c|c|}
\hline
$|\ell,m\rangle$ & twist-2 & twist-3 \\\hline\hline
$|1,1\rangle$ & $\sin\theta\sin\left(\phi_h-\phi_R\right)$ & $\sin\theta\sin\phi_R$ \\\hline
$|1,0\rangle$ & 0 & $\cos\theta\sin\phi_h$ \\\hline
$|1,-1\rangle$ & $~$ & $\sin\theta\sin\left(2\phi_h-\phi_R\right)$ \\\hline
\end{tabular}
~\\~\\~\\
\begin{tabular}{|c|c|c|}
\hline
$|\ell,m\rangle$ & twist-2 & twist-3 \\\hline\hline
$|2,2\rangle$ & $\sin^2\theta\sin\left(2\phi_h-2\phi_R\right)$ & $\sin^2\theta\sin\left(-\phi_h+2\phi_R\right)$ \\\hline
$|2,1\rangle$ & $\sin\theta\cos\theta\sin\left(\phi_h-\phi_R\right)$ & $\sin\theta\cos\theta\sin\phi_R$ \\\hline
$|2,0\rangle$ & 0 & $\left(3\cos^2\theta-1\right)\sin\phi_h$ \\\hline
$|2,-1\rangle$ & $~$ & $\sin\theta\cos\theta\sin\left(2\phi_h-\phi_R\right)$ \\\hline
$|2,-2\rangle$ & $~$ & $\sin^2\theta\sin\left(3\phi_h-2\phi_R\right)$ \\\hline
\end{tabular}
~\\~\\
\end{center}
The inner product of two functions $f,g\in S$ is
\begin{equation}
\langle fg\rangle=
\int_{-\pi}^{\pi}d\phi_h
\int_{-\pi}^{\pi}d\phi_R
\int_{0}^{\pi}d\theta
f\left(\phi_h,\phi_R,\theta\right)
g\left(\phi_h,\phi_R,\theta\right)
\end{equation}
If $\langle fg\rangle=0$, then $f$ is orthogonal to $g$. Normalization of the
functions can be obtained by $f\mapsto f/\sqrt{\langle ff\rangle}$, which
ensures $\langle ff\rangle=1$.  

This inner product can be weighted by the yield data, to see how the
orthogonality between each $f$ and $g\in S$ is affected by acceptance
limitations.  Let $D$ represent a histogram of the yield in bins of $\phi_h$,
$\phi_R$, and $\theta$.  The contents of each bin $h$ of $\phi_h$, $r$ of
$\phi_R$ and $t$ of $\theta$ is denoted $D_{hrt}$. Let $w_h,w_r,w_t$ be the bin
widths.  The above inner product can be weighted by the yield, expressing
$\langle fg\rangle$ with the $\phi_h$, $\phi_R$, and $\theta$ domains
discretized:
\begin{equation}
\displaystyle
\langle fg\rangle=
%\frac{1}{\sqrt{\langle ff\rangle\langle gg\rangle}}
N_fN_g
\sum_h\sum_r\sum_t
w_hw_rw_tD_{hrt}f_{hrt}g_{hrt}
\end{equation}
where the normalization factors $N_f$ for $f\in S$ are given by
\begin{equation}
N_f=
\left[
\sum_h\sum_r\sum_t
w_hw_rw_tD_{hrt}f_{hrt}f_{hrt}
\right]^{-1/2}
\end{equation}
to ensure $\langle ff\rangle=1$. The sums run over all the bins in the
discretized space.

\subsection*{Impact of non-orthogonal modulations on asymmetry amplitudes}
An asymmetry $A$ can be expanded in terms of functions of $S$, i.e., amplitudes
$A_k$ of modulations $f_k\in S$, as
\begin{equation}
A=\sum_k{A_kf_k}
\end{equation}
But since the $f_k$ are not linearly independent, each $f_k$ can be expanded
as 
\begin{equation}
f_k=\sum_i{\langle f_if_k\rangle f_i}
\end{equation}
where in general $\langle f_if_k\rangle=\delta_{ik}+\left(1-\delta_{ik}\right)c$
for some $c\in(-1,1)$. The asymmetry thus
expands as
\begin{equation}
A=\sum_k{A_k\sum_i{\langle f_if_k\rangle f_i}}
\end{equation}
Pulling out the $i=k$ terms gives
\begin{equation}
A=\sum_k\left[A_kf_k+\sum_{i\neq k}{A_k\langle f_if_k\rangle f_i}
\right]
\end{equation}
which can be rearranged to
\begin{equation}
A=\sum_k\left[A_k+\sum_{i\neq k}{A_i\langle f_kf_i\rangle}
\right]f_k
\end{equation}

Consider the example of two modulations $f_1$ and $f_2$.
If $f_1$ and $f_2$ are orthogonal, then the amplitude of $f_1$ is $A_1$ and the
amplitude of $f_2$ is $A_2$.  If they are not orthogonal, 
the amplitude of $f_1$ is shifted to $A_1+\langle f_1f_2\rangle A_2$,
and vice versa for the amplitude of $f_2$.

Let $\left\{A_k'\right\}$ be the set of true physics amplitudes (or injected
amplitudes in the Monte Carlo).  The amplitudes $\left\{A_k\right\}$ that will result
from a fit should be 
\begin{equation}
A_k=A_k'+\sum_{i\neq k}{A_i'\langle f_kf_i\rangle}
\label{eqM}
\end{equation}
These two sets $\left\{A_k'\right\}$ and $\left\{A_k\right\}$ of amplitudes can
respectively be written as vectors $A'$ and $A$. Let $O$ be the matrix with
elements $O_{ij}=\langle f_if_j\rangle$. Then equation \ref{eqM} can be written
as $A=OA'$. Assuming $O$ is invertible, the true asymmetry amplitudes are thus
$A'=O^{-1}A$.


\section*{Useful tables}
\begin{center}
\begin{tabular}{|c|c|c|}
\hline
$m$ & twist-2 & twist-3 \\\hline\hline
$2$ & $\sin\left(2\phi_h-2\phi_R\right)$ & $\sin\left(-\phi_h+2\phi_R\right)$ \\\hline
$1$ & $\sin\left(\phi_h-\phi_R\right)$ & $\sin\left(\phi_R\right)$ \\\hline
$0$ & $0$ & $\sin\left(\phi_h\right)$ \\\hline
$-1$ & $~$ & $\sin\left(2\phi_h-\phi_R\right)$ \\\hline
$-2$ & $~$ & $\sin\left(3\phi_h-2\phi_R\right)$ \\\hline
\end{tabular}
\end{center}




\end{document} 
