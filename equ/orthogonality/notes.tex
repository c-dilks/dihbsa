\documentclass[12pt]{article}
\usepackage{setspace} % double space
\onehalfspacing % 1.5 space
\usepackage[left=1in,top=1in,right=1in,bottom=1in,nohead,nofoot]{geometry} %margins
\setlength\footskip{0.25in} % separate page number from last line
\setlength\parindent{0.5in} % indent paragraphs; first paragraphs are not indented
\usepackage{epstopdf}

% begin upright mu (\umu) definition
%\usepackage{lmodern}
%\usepackage{textcomp}
%\DeclareFontFamily{U}{euc}{}
%\DeclareFontShape{U}{euc}{m}{n}{<-6>eurm5<6-8>eurm7<8->eurm10}{}
%\DeclareSymbolFont{AMSc}{U}{euc}{m}{n}
%\DeclareMathSymbol{\umu}{\mathord}{AMSc}{"16}
% end upright mu


\usepackage{amssymb}
\usepackage{amsmath}
%\usepackage{MnSymbol} % this must be defined after \umu (upright mu)

\begin{document}

\begin{center}
{\LARGE 
Orthogonality of Modulations
}
\end{center}

\subsection*{Cross section modulations}


The goal is to assess the orthogonality of the various modulations of the beam
spin asymmetry, or in general, of the polarization-dependent cross section. Let
$S$ be the set of such modulation functions, dependent on $\phi_h$, $\phi_R$,
and $\theta$. The functions in $S$ are preferred to be mutually orthogonal,
since the measurement of an amplitude of one modulation $f$ can be impacted by
the amplitudes of other modulations which are not orthogonal to $f$.  If the
yield data do not cover the full $\phi_h,\phi_R,\theta$ range, the orthogonality
of the functions of $S$ can be impacted, as the integration ranges will be
limited by the acceptance. 

For the SIDIS dihadron production cross section,
the functions of $S$ are written as the product of two functions, 
$P_{\ell}^m\Phi_t^m$, which depend
on the angular momentum state $|\ell,m\rangle_t$ of the dihadron system, where
$\ell$ and $m$ are the angular momentum eigenvalues and $t$ is the twist.
Written explicitly, the functions are:
\begin{itemize}
\item $\Phi_t^m\left(\phi_h,\phi_R\right)$: 
azimuthal modulations, which depend on $m$ and twist $t$. These functions are
elements of the 2-dimensional Fourier basis. For a longitudinally polarized electron beam 
and an unpolarized proton target, the functions are
  \begin{itemize}
  \item twist 2: $\Phi_2^m=\sin\left[m\left(\phi_h-\phi_R\right)\right]$
  \item twist 3: $\Phi_3^m=\sin\left[\left(1-m\right)\phi_h+m\phi_R\right]$
  \end{itemize}
\item $P_{\ell}^{m}\left(\cos\theta\right)$:
associated Legendre polynomials, which depend on the angular momentum
eigenstate; note that $P_{\ell}^{-m}=P_{\ell}^{m}$ is assumed. This is the
partial wave expansion of the DiFFs in $\cos\theta$.
\end{itemize}
The functions for $\ell\in\{0,1,2\}$ are (up to normalization factors):
\begin{center}
\begin{tabular}{|c|c|c|}
\hline
$|\ell,m\rangle$ & twist-2 & twist-3 \\\hline\hline
$|0,0\rangle$ & 0 & $\sin\phi_h$ \\\hline
\end{tabular}
~ ~ ~
\begin{tabular}{|c|c|c|}
\hline
$|\ell,m\rangle$ & twist-2 & twist-3 \\\hline\hline
$|1,1\rangle$ & $\sin\theta\sin\left(\phi_h-\phi_R\right)$ & $\sin\theta\sin\phi_R$ \\\hline
$|1,0\rangle$ & 0 & $\cos\theta\sin\phi_h$ \\\hline
$|1,-1\rangle$ & $~$ & $\sin\theta\sin\left(2\phi_h-\phi_R\right)$ \\\hline
\end{tabular}
~\\~\\~\\
\begin{tabular}{|c|c|c|}
\hline
$|\ell,m\rangle$ & twist-2 & twist-3 \\\hline\hline
$|2,2\rangle$ & $\sin^2\theta\sin\left(2\phi_h-2\phi_R\right)$ & $\sin^2\theta\sin\left(-\phi_h+2\phi_R\right)$ \\\hline
$|2,1\rangle$ & $\sin\theta\cos\theta\sin\left(\phi_h-\phi_R\right)$ & $\sin\theta\cos\theta\sin\phi_R$ \\\hline
$|2,0\rangle$ & 0 & $\left(3\cos^2\theta-1\right)\sin\phi_h$ \\\hline
$|2,-1\rangle$ & $~$ & $\sin\theta\cos\theta\sin\left(2\phi_h-\phi_R\right)$ \\\hline
$|2,-2\rangle$ & $~$ & $\sin^2\theta\sin\left(3\phi_h-2\phi_R\right)$ \\\hline
\end{tabular}
~\\~\\
\end{center}


\subsection*{Testing orthogonality under acceptance constraints}

The inner product of two functions $f,g\in S$ is
\begin{equation}
\langle f,g\rangle=
\int_{-\pi}^{\pi}d\phi_h
\int_{-\pi}^{\pi}d\phi_R
\int_{0}^{\pi}d\theta
f\left(\phi_h,\phi_R,\theta\right)
g\left(\phi_h,\phi_R,\theta\right)
\label{eqIP}
\end{equation}
If $\langle f,g\rangle=0$, then $f$ is orthogonal to $g$. Normalization of the
functions can be obtained by $f\mapsto f/\sqrt{\langle ff\rangle}$, which
ensures $\langle ff\rangle=1$.  
Before continuing, denote a point $\left(\phi_h,\phi_R,\theta\right)$ by $\phi$,
to simplify the notation; any integral over $\phi$ is definite, as in equation
\ref{eqIP}.

This inner product can be weighted by the acceptance, denoted $y(\phi)$, to see how the
orthogonality between each $f$ and $g\in S$ is affected by acceptance
limitations. The weighted inner product, denoted $O$,
is a functional of $y(\phi)$, $f(\phi)$, and $g(\phi)$:
\begin{equation}
O\left[f,g,y\right]=\langle yf,g\rangle=\int{d\phi y(\phi)f(\phi)g(\phi)}
\label{eqWPI}
\end{equation}
The acceptance $y(\phi)$ can be estimated using Monte Carlo techniques, and be
modelled as a Fourier expansion. This model should be constrained to within each
kinematic bin used for asymmetry estimation, because $y(\phi)$ can depend on the
kinematics: for example, $y(\phi)$ may be different at low $M_h$ compared to
high $M_h$.

In practice, it is easier to use the yield distribution $Y(b_\phi)$ as a proxy for the
acceptance $y(\phi)$, where $b_\phi$ is a single bin in $\phi$. The yield does
not need to be normalized, since we can always normalize $O$ itself. Since
$Y(b_\phi)$ is a discrete function, the integral in equation \ref{eqWPI} needs
to be replaced by a sum. Let $f(b_\phi)$ and $g(b_\phi)$ be $f$ and $g$
evaluated at the bin center of bin $b_\phi$. With $v(b_\phi)$ denoting the volume of
bin $b_\phi$, the discrete form of equation \ref{eqWPI} becomes:
\begin{equation}
\displaystyle
O[f,g,Y]=N_fN_g\sum_{b_\phi}{
  v(b_\phi)Y(b_\phi)f(b_\phi)g(b_\phi)
}
\end{equation}
where the sum runs over all bins in $\phi$.
The normalization factors $N_f$ for $f\in S$ are determined by enforcing
$O[f,f,Y]=1$:
\begin{equation}
\displaystyle
N_f=\left(\sum{vYf^2}\right)^{-1/2}
\end{equation}

Next, we consider the uncertainty propagation. Let $\sigma_O[f,g,Y]$ denote the
uncertainty on $O[f,g,Y]$, which is dominantly from the statistical uncertainty
on the yield, $\sigma_Y(b_\phi)$. With the Poissonian statistics assumption,
$\sigma_Y(b_\phi)=\sqrt{Y(b_\phi)}$. Assume the uncertainty on $\phi$ is
negligible in this context, and that there is no uncertainty from the functions
$f$ and $g$, as they are analytically defined. We thus have:
\begin{equation}
\sigma_O[f,g,Y]=N_fN_g\sum{v\sqrt{Y}fg}=O\left[f,g,\sqrt{Y}\right]
\end{equation}


\subsection*{Impact of non-orthogonal modulations on asymmetry amplitudes}
From here on, let $\langle fg\rangle:=O[f,g,Y]$.
An asymmetry $A$ can be expanded in terms of functions of $S$, i.e., amplitudes
$A_k$ of modulations $f_k\in S$, as
\begin{equation}
A=\sum_k{A_kf_k}
\end{equation}
wherein each $f_k$ can be expanded as 
\begin{equation}
f_k=\sum_i{\langle f_if_k\rangle f_i}
\end{equation}
The asymmetry thus expands as
\begin{equation}
A=\sum_k{A_k\sum_i{\langle f_if_k\rangle f_i}}
\end{equation}
Pulling out the $i=k$ terms gives
\begin{equation}
A=\sum_k\left[A_kf_k+\sum_{i\neq k}{A_k\langle f_if_k\rangle f_i}
\right]
\end{equation}
which can be rearranged to
\begin{equation}
A=\sum_k\left[A_k+\sum_{i\neq k}{A_i\langle f_kf_i\rangle}
\right]f_k
\end{equation}

Consider the example of two modulations $f_1$ and $f_2$.
If $f_1$ and $f_2$ are orthogonal, then the amplitude of $f_1$ is $A_1$ and the
amplitude of $f_2$ is $A_2$.  If they are not orthogonal, 
the amplitude of $f_1$ is shifted to $A_1+\langle f_1f_2\rangle A_2$,
and vice versa for the amplitude of $f_2$.

Let $\left\{A_k'\right\}$ be the set of true physics amplitudes (or injected
amplitudes in the Monte Carlo).  The amplitudes $\left\{A_k\right\}$ that will result
from a fit should be 
\begin{equation}
A_k=A_k'+\sum_{i\neq k}{A_i'\langle f_kf_i\rangle}
\label{eqM}
\end{equation}
These two sets $\left\{A_k'\right\}$ and $\left\{A_k\right\}$ of amplitudes can
respectively be written as vectors $A'$ and $A$. Let $O$ be the matrix with
elements $O_{ij}=\langle f_if_j\rangle$. Then equation \ref{eqM} can be written
as $A=OA'$. Assuming $O$ is invertible, the true asymmetry amplitudes are thus
$A'=O^{-1}A$.


\section*{Useful tables}
\begin{center}
\begin{tabular}{|c|c|c|}
\hline
$m$ & twist-2 & twist-3 \\\hline\hline
$2$ & $\sin\left(2\phi_h-2\phi_R\right)$ & $\sin\left(-\phi_h+2\phi_R\right)$ \\\hline
$1$ & $\sin\left(\phi_h-\phi_R\right)$ & $\sin\left(\phi_R\right)$ \\\hline
$0$ & $0$ & $\sin\left(\phi_h\right)$ \\\hline
$-1$ & $~$ & $\sin\left(2\phi_h-\phi_R\right)$ \\\hline
$-2$ & $~$ & $\sin\left(3\phi_h-2\phi_R\right)$ \\\hline
\end{tabular}
\end{center}




\end{document} 
